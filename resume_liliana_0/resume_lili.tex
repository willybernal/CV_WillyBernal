
%%%%%%%%%%%%%%%%%%%%%%%%%%%%%%%%%%%%%%%%%%%%%%%%%%%%%%%%%%%%%%%%%%%%%%%%
%%%%%%%%%%%%%%%%%%%%%% Simple LaTeX CV Template %%%%%%%%%%%%%%%%%%%%%%%%
%%%%%%%%%%%%%%%%%%%%%%%%%%%%%%%%%%%%%%%%%%%%%%%%%%%%%%%%%%%%%%%%%%%%%%%%
 
%%%%%%%%%%%%%%%%%%%%%%%%%%%%%%%%%%%%%%%%%%%%%%%%%%%%%%%%%%%%%%%%%%%%%%%%
%% NOTE: If you find that it says                                     %%
%%                                                                    %%
%%                           1 of ??                                  %%
%%                                                                    %%
%% at the bottom of your first page, this means that the AUX file     %%
%% was not available when you ran LaTeX on this source. Simply RERUN  %%
%% LaTeX to get the ``??'' replaced with the number of the last page  %%
%% of the document. The AUX file will be generated on the first run   %%
%% of LaTeX and used on the second run to fill in all of the          %%
%% references.                                                        %%
%%%%%%%%%%%%%%%%%%%%%%%%%%%%%%%%%%%%%%%%%%%%%%%%%%%%%%%%%%%%%%%%%%%%%%%%

%%%%%%%%%%%%%%%%%%%%%%%%%%%% Document Setup %%%%%%%%%%%%%%%%%%%%%%%%%%%%

% Don't like 10pt? Try 11pt or 12pt
\documentclass[10pt]{article}

% This is a helpful package that puts math inside length specifications
\usepackage{calc}
\usepackage{verbatim} 
% Simpler bibsection for CV sections
% (thanks to natbib for inspiration)
\makeatletter
\newlength{\bibhang}
\setlength{\bibhang}{1em}
\newlength{\bibsep}
 {\@listi \global\bibsep\itemsep \global\advance\bibsep by\parsep}
\newenvironment{bibsection}
    {\minipage[t]{\linewidth}\list{}{%
        \setlength{\leftmargin}{\bibhang}%
        \setlength{\itemindent}{-\leftmargin}%
        \setlength{\itemsep}{\bibsep}%
        \setlength{\parsep}{\z@}%
        }}
    {\endlist\endminipage}
\makeatother

% Layout: Puts the section titles on left side of page
\reversemarginpar

%
%         PAPER SIZE, PAGE NUMBER, AND DOCUMENT LAYOUT NOTES:
%
% The next \usepackage line changes the layout for CV style section
% headings as marginal notes. It also sets up the paper size as either
% letter or A4. By default, letter was used. If A4 paper is desired,
% comment out the letterpaper lines and uncomment the a4paper lines.
%
% As you can see, the margin widths and section title widths can be
% easily adjusted.
%
% ALSO: Notice that the includefoot option can be commented OUT in order
% to put the PAGE NUMBER *IN* the bottom margin. This will make the
% effective text area larger.
%
% IF YOU WISH TO REMOVE THE ``of LASTPAGE'' next to each page number,
% see the note about the +LP and -LP lines below. Comment out the +LP
% and uncomment the -LP.
%
% IF YOU WISH TO REMOVE PAGE NUMBERS, be sure that the includefoot line
% is uncommented and ALSO uncomment the \pagestyle{empty} a few lines
% below.
%

%% Use these lines for letter-sized paper
\usepackage[paper=letterpaper,
            %includefoot, % Uncomment to put page number above margin
            marginparwidth=1.2in,     % Length of section titles
            marginparsep=.05in,       % Space between titles and text
            margin=1in,               % 1 inch margins
            includemp]{geometry}

%% Use these lines for A4-sized paper
%\usepackage[paper=a4paper,
%            %includefoot, % Uncomment to put page number above margin
%            marginparwidth=30.5mm,    % Length of section titles
%            marginparsep=1.5mm,       % Space between titles and text
%            margin=25mm,              % 25mm margins
%            includemp]{geometry}

%% More layout: Get rid of indenting throughout entire document
\setlength{\parindent}{0in}

%% This gives us fun enumeration environments. compactitem will be nice.
\usepackage{paralist}

%% Reference the last page in the page number
%
% NOTE: comment the +LP line and uncomment the -LP line to have page
%       numbers without the ``of ##'' last page reference)
%
% NOTE: uncomment the \pagestyle{empty} line to get rid of all page
%       numbers (make sure includefoot is commented out above)
%
\usepackage{fancyhdr,lastpage}
\pagestyle{fancy}
%\pagestyle{empty}      % Uncomment this to get rid of page numbers
\fancyhf{}\renewcommand{\headrulewidth}{0pt}
\fancyfootoffset{\marginparsep+\marginparwidth}
\newlength{\footpageshift}
\setlength{\footpageshift}
          {0.5\textwidth+0.5\marginparsep+0.5\marginparwidth-2in}
\lfoot{\hspace{\footpageshift}%
       \parbox{4in}{\, \hfill %
                    \arabic{page} of \protect\pageref*{LastPage} % +LP
%                    \arabic{page}                               % -LP
                    \hfill \,}}

% Finally, give us PDF bookmarks
\usepackage{color,hyperref}
\definecolor{darkblue}{rgb}{0.0,0.0,0.3}
\hypersetup{colorlinks,breaklinks,
            linkcolor=darkblue,urlcolor=darkblue,
            anchorcolor=darkblue,citecolor=darkblue}

%%%%%%%%%%%%%%%%%%%%%%%% End Document Setup %%%%%%%%%%%%%%%%%%%%%%%%%%%%


%%%%%%%%%%%%%%%%%%%%%%%%%%% Helper Commands %%%%%%%%%%%%%%%%%%%%%%%%%%%%

% The title (name) with a horizontal rule under it
%
% Usage: \makeheading{name}
%
% Place at top of document. It should be the first thing.
\newcommand{\makeheading}[1]%
        {\hspace*{-\marginparsep minus \marginparwidth}%
         \begin{minipage}[t]{\textwidth+\marginparwidth+\marginparsep}%
                {\large \bfseries #1}\\[-0.15\baselineskip]%
                 \rule{\columnwidth}{1pt}%
         \end{minipage}}

% The section headings
%
% Usage: \section{section name}
%
% Follow this section IMMEDIATELY with the first line of the section
% text. Do not put whitespace in between. That is, do this:
%
%       \section{My Information}
%       Here is my information.
%
% and NOT this:
%
%       \section{My Information}
%
%       Here is my information.
%
% Otherwise the top of the section header will not line up with the top
% of the section. Of course, using a single comment character (%) on
% empty lines allows for the function of the first example with the
% readability of the second example.
\renewcommand{\section}[2]%
        {\pagebreak[2]\vspace{1.3\baselineskip}%
         \phantomsection\addcontentsline{toc}{section}{#1}%
         \hspace{0in}%
         \marginpar{
         \raggedright \scshape #1}#2}

% An itemize-style list with lots of space between items
\newenvironment{outerlist}[1][\enskip\textbullet]%
        {\begin{itemize}[#1]}{\end{itemize}%
         \vspace{-.6\baselineskip}}

% An environment IDENTICAL to outerlist that has better pre-list spacing
% when used as the first thing in a \section
\newenvironment{lonelist}[1][\enskip\textbullet]%
        {\vspace{-\baselineskip}\begin{list}{#1}{%
        \setlength{\partopsep}{0pt}%
        \setlength{\topsep}{0pt}}}
        {\end{list}\vspace{-.6\baselineskip}}

% An itemize-style list with little space between items
\newenvironment{innerlist}[1][\enskip\textbullet]%
        {\begin{compactitem}[#1]}{\end{compactitem}}

% To add some paragraph space between lines.
% This also tells LaTeX to preferably break a page on one of these gaps
% if there is a needed pagebreak nearby.
\newcommand{\blankline}{\quad\pagebreak[2]}

% 

%%%%%%%%%%%%%%%%%%%%%%%% End Helper Commands %%%%%%%%%%%%%%%%%%%%%%%%%%%

%%%%%%%%%%%%%%%%%%%%%%%%% Begin CV Document %%%%%%%%%%%%%%%%%%%%%%%%%%%%

\begin{document}
\makeheading{Liliana E.~Ferrua Villanueva}

\section{Contact Information}
%
% NOTE: Mind where the & separators and \\ breaks are in the following
%       table.
%
% ALSO: \rcollength is the width of the right column of the table
%       (adjust it to your liking; default is 1.85in).
%
\newlength{\rcollength}\setlength{\rcollength}{1.85in}%
%
\begin{tabular}[t]{@{}p{\textwidth-\rcollength}p{\rcollength}}
\href{http://www.ece.osu.edu/}%
     {Liliana E Ferrua Villanueva} 					& \\
%\href{http://www.upenn.edu/}{The University of Pennsylvania}
Calle Salamanca 216              & \textit{Cell:} (51) 976-739-876\\
Pueblo Libre, Lima 21 Peru	\\
																& \textit{E-mail:}
\href{mailto:lilianaferrua@gmail.com}{lilianaferrua@gmail.com}\\
%& %\textit{WWW:}
%\href{http://www.tedpavlic.com/}{www.tedpavlic.com}\\
\end{tabular}

%\section{Citizenship}
%
%Peruvian

\section{Research Interests}
%
Textile Engineering, Engineering Management, Fashion Merchandising 

\section{Education}
%
\href{http://www.uni.edu.pe}{\textbf{National University of Engineering - uni.edu.pe}},
Lima, Peru
\begin{outerlist}

\item[] B.S.,
        \href{http://www.uni.edu.pe}
             {Textile Engineering}, (May 2010)
        \begin{innerlist}
        \item \emph{Valedictoriam}, With Honors in Engineering
        \item Electrical specialization (emphasis on Embedded Systems)
        \end{innerlist}
        
\end{outerlist}

\href{http://www.upenn.edu/}{\textbf{College of Engineers of Peru 2011}},
Lima, Peru
\begin{outerlist}

\item[] Specialized studies,
        \href{http://www.ese.upenn.edu/}
             {INtegrated Management Systesm for the Quality, Environment, Occupational Helath , Saferty and Social Responsibility}, (2011)
        \begin{innerlist}
        \item Diploma
        \end{innerlist}
        
\end{outerlist}


\section{Awards}
%
\textbf{Best Demo Award} %\href{http://www.buildsys.org/2012/}{BuildSys '12, Toronto, Canada.}
	\begin{innerlist}
		\item \href{http://www.buildsys.org/2012/}{4th ACM Workshop on Embedded Sensing Systems For Energy-Efficiency in Buildings} for \href{http://repository.upenn.edu/mlab_papers/51/}{\textbf{MLE+: A Tool for Integrated Design and Deployment of Energy-Efficient Building Controls.}}\\
	\end{innerlist}

\href{http://www.upenn.edu/}{\textbf{The University of Pennsylvania}}
	\begin{innerlist}
		\item Dean's List,\\
        	2005-2006, 2007-2008. 
	\end{innerlist}

\section{Current Projects}
%
\textbf{MLE+: A Tool for Integrated Design and Deployment of Energy-Efficient Buildings}
\begin{innerlist}
\item \href{http://www.upenn.edu/}{MLE+} is a Co-Simulation Toolbox for integrated design and deployment of energy-efficient building controls for buildings simulated in EnergyPlus. MLE+ leverages the high-fidelity building simulation capabilities of EnergyPlus and the scientific computation and controller design capabilities of Matlab. 
\item The software provides integrated building simulation and controller formulation with integrated support for system identification, control design, optimization, simulation analysis and communication between software applications and real building equipment. (\href{http://mlab.seas.upenn.edu/mlep}{mlab.seas.upenn.edu/mlep})\\
\end{innerlist}
%
\section{Past Projects}
%
\textbf{Home Automation Network}
\begin{innerlist}
\item A real-time, low-power wireless sensor network system that can actuate any AC appliance, open and close window blinds, and monitor, in real-time, power consumption of each device on the network. 
\item The project implements the \href{http://www.nanork.org/wiki/FireFly}{Firefly} sensor nodes, \href{http://www.nanork.org}{Nano-RK}(a realtime operating system), relays, and various other electronic components to build the hardware for actuation. 
\item The \href{http://www.nanork.org/wiki/FireFly}{Firefly} sensor nodes wirelessly (IEEE 802.15.4) communicate with a gateway node while an iPhone web application and Java web application facilitates two-way communication with the actuation network over Wi-Fi or 3G and provides an interface for the user for actuation and sensing. \\
\end{innerlist}

\textbf{Electrocardiogram Wireless Sensor, (\href{http://mlab.seas.upenn.edu/zipcare}{\textbf{iBOD}})}
\begin{innerlist}
\item This project consists of a high-confidence and low profile medical device for long-term onbody monitoring. 
\item This project targets low cost disposable on-body hardware-base health-strip, an adaptive real-time operating system design for runtime programmable control and long-term context-based medical sensor data interpretation. \\
\end{innerlist}

%\newpage
\section{Work Experience}
%%------------UFITEC----------------%%
\href{http://www.upenn.edu}{\textbf{UFITEC S.A.C.}}
\hfill \textbf{Lima, Peru}
\begin{outerlist}
\item[] \textit{Production Assistant}
    \hfill \textbf{November 2007 - August 2008}
    \begin{innerlist}
        \item Design, implementation and evaluation of improvement plans to increase textile production in the company.    
        \item Design of advanced controls for Energy-Efficient Buildings. 
		\item Design and construction of Wireless Sensor Networks for data gathering.  
    \end{innerlist}
\end{outerlist}

%%------------BELCORP----------------%%
\href{http://www.upenn.edu}{\textbf{BELCORP}}
\hfill \textbf{Lima, Peru}
\begin{outerlist}
\item[] \textit{Quality Mangement Assistant}
    \hfill \textbf{December 2010 - April 2012}
    \begin{innerlist}
        \item Research and establish and assurance of standard methodologies and processes for management and development.        
        \item Garment and Fabric parameters analysis (shrinkage, twisting, weight, appearance, strength tests).
		\item Quality control process of patterns and fabrics.  
    \end{innerlist}
\end{outerlist}
%%------------BELCORP----------------%%


%%------------BELCORP----------------%%

%% 
\section{Technical Skills}
%
\textbf{Building and Mechanical Simulation Software:}
\begin{innerlist}
	\item EnergyPlus, Design Builder, Rhinoceros, Daysim, Radiance, Open Studio, COMSOL, Fluent, Ecotect, SolidWorks.\\   
\end{innerlist}

\textbf{Embedded Systems:} 
\begin{innerlist}
	\item Entensive Hardware and software experience in embedded systems, Real Time operating systems, wireless cards (Chipcon CC2420), and analog and digital electronics. \href{http://www.nanork.org/wiki/FireFly}{Firefly}, \href{http://modlabupenn.org/ckbot/}{CKbot}. ARM microprocessors (\href{http:www.mbed.org}{mbed}), Motorola MCU's, Texas Instruments MCU's, Atmel ATmega MCU's, Microchip PIC MCU's, and others).
	\item Real-Time Operating Systems: \href{http:www.nano-rk.org}{Nano-RK}\\
\end{innerlist}

\textbf{Programming Experience:}
\begin{innerlist}
	\item C/C++, Java, Matlab, HTML, Python, CSS. \\
\end{innerlist}

\textbf{Information Technology}
\begin{innerlist}
	\item Networking (UDP,TCP, SLIPstream), Service (Apache).\\
\end{innerlist}

\textbf{Analog and Digital Electronics}
\begin{innerlist}
	\item Analog and Digital Electronics: Bipolar and FET implementations of continuous and switched amplifiers, modulators, and filters.\\ \end{innerlist}

\textbf{Computer-Aided Design:}
\begin{innerlist}
	\item Cadence OrCAD, NI Multisim, SPICE, Eagle CADsoft, AutoCAD, SolidWorks, GoogleSkepup.\\
\end{innerlist}

\textbf{Matlab}
\begin{innerlist}
    \item Experience with the following packages: Linear Algebra, Fourier transforms, Nonlinear Numerical Methods, Support Vector Machines, GUI utilities, Optimization, Communication tools, Visualization, Simulink, MPC toolbox.\\
\end{innerlist}

\textbf{Engineering Expertise}
\begin{innerlist}
	\item Control: Linear Systems Theory, Feedback, Non-Linear Control and Optimal Control Theory.
	\item Optimization: Linear Optimization, Convex Optimization. 
	\item Robotics: Machine Perception, Motion Planning.
	\item Statistics: Support Vector Machines, Regression Analysis, Estimation.\\
\end{innerlist}

\section{Affiliations}
%
\href{http://www.upenn.edu}{\textbf{The University of Pennsylvania}}
\begin{innerlist}
\item \href{http://www.seas.upenn.edu/~tbp/}{Tau Beta Pi}, Engineering Honor Society, Delta Chapter.
\item \href{http://www.seas.upenn.edu/~ekn/}{Eta Kappa Nu}, Electrical and Computer Engineering Honor Society. Lambda Chapter. 
\item \href{http://www.seas.upenn.edu/~shpe/}{SPHE}, Society of Hispanic Professional Engineers
\end{innerlist}

\section{References}
Nelly Canepa Moscoso
\hfill \textbf{Allpa S.A.C.}\\
Operations Manager / General Director\\
\href{mailto:ncanepa@allpa.com.pe}{ncanepa@allpa.com.pe}\\

Liliana Granados Rutty
\hfill \textbf{Belcorp}\\
Head of the Quality Management Department\\
\href{mailto:lgranados@belcorp.biz}{lgranados@belcorp.biz}\\ 

Master Martin Reaño
\hfill \textbf{Sociedad Nacional de Industrias}\\
Textile Committee Manager\\ 
\href{mailto:mreano@sni.org.pe}{mreano@sni.org.pe}\\

Engineer Segundo Octavio Arroyo Gastelu
\hfill \textbf{Sociedad Nacional de Industrias}\\
Textile Consultant-Spinning Department\\
\href{mailto:octavios_1904@hotmail.com}{octavios\_19104@hotmail.com}\\
\end{document}

%%%%%%%%%%%%%%%%%%%%%%%%%% End CV Document %%%%%%%%%%%%%%%%%%%%%%%%%%%%%
